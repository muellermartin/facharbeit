% Papierformat
\documentclass[a4paper, 12pt, bibtotocnumbered, liststotocnumbered]{scrartcl}

% Seitenränder
\usepackage[top=2.5cm, left=4cm, right=2cm, bottom=2.5cm]{geometry}

% Zeichenkodierung
\usepackage[utf8]{inputenc}

% Deutsche Silbentrennung
\usepackage[ngerman]{babel}

% Deutsches Literaturverzeichnis
\usepackage{babelbib}

% Zitieren mit cite-key ermöglichen
\usepackage{cite}

% Erweiterte Funktionialität von URLs
\usepackage{url}

\begin{document}
	\title{Entwicklung eines Remote-Heads für einen Kamerakran}
	\author{Martin Müller}

	% Titelseite
	\begin{center}
		\large{Clavius-Gymnasium Bamberg}\\
		\Large{Facharbeit im Leistungskurs Phsyik}\\
		\large{Kollegstufe 2010/2011}
		\vfill{Entwicklung eines Remote-Heads für einen Kamerakran}
	\end{center}

	\pagebreak

	% Inhaltsverzeichnis
	\tableofcontents

	\pagebreak

	% Inhalt
	\section{Einleitung}
	Blabla

	\pagebreak

	% Bibliographie
	% Um die Bibliographie einzubinden sind folgende Schritte notwendig:
	% 1. LaTeX setzen (findet NOTWENDIGE Verweise und speichert diese in .aux-Datei)
	% 2. BibTeX setzen (generiert .bbl-Datei)
	% 3. LaTeX erneut setzen (bindet .bbl-Datei in das Dokument ein)
	\nocite{*}
	\bibliography{Quellen}
	\bibliographystyle{alphadin}

	% Bilderverzeichnis
	\listoffigures

	\pagebreak

	% Erklärung
	\section*{Erklärung}
	Ich erkläre, dass ich die Facharbeit ohne fremde Hilfe angefertigt und nur die im Literaturverzeichnis angeführten Quellen und Hilfsmittel benutzt habe.
	\\
	\\
	\\
	\begin{minipage}{3cm}
		..................................,\\
		\it{Ort}
	\end{minipage}
	\hspace{1.5cm}
	\begin{minipage}{2cm}
		...........................\\
		\it{Datum}
	\end{minipage}
	\hspace{1.5cm}
	\begin{minipage}{8cm}
		.........................................................\\
		\it{Unterschrift des Schülers}
	\end{minipage}
\end{document}